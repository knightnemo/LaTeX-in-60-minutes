\documentclass[a4paper,12pt]{article}
\usepackage{color}
\usepackage{graphicx}
\usepackage{epigraph}
\usepackage{amssymb}%这个包引入特殊字符
\usepackage{amsmath}%这个包引入矩阵etc.
\begin{document}

\title{My First \LaTeX document
\\aka. \\A Beginner's Guide to \LaTeX}
\author{KnightNemo@THSS}
\date{\today}
\maketitle
\newpage

\definecolor{mypurple}{RGB}{102,0,153}

\pagenumbering{roman}
\tableofcontents
\newpage
\pagenumbering{arabic}

\begin{quotation}
    \centering {\huge Welcome to \LaTeX.} 
\end{quotation}
\epigraph{ Science is what we understand well enough to explain to a computer, Art is all the rest. }
{\textit{Donald Knuth}}

\section{Introduction}
    This is the Introduction.

    \LaTeX is a widely-used software system for typesetting documents.
    blah blah blah\dots
    In short, it is an awesome way to
    write well-structured and fully-in-control documents.

    Anyway, as you have came this far so as to discover this doc,
    I may as well cut out the piffle. 
    So, Let's plunge right in!

\section{Font-Related}

    \subsection{Diverse Font Styles}
    \label{sec1}
    There are a variety of text styles in \LaTeX: 

        \textit{words in italics}

        \textsl{words slanted}

        \textsc{words in smallcaps}

        \textbf{words in bold}

        \texttt{words in teletype}

        \textsf{sans serif words}

        \textrm{roman words}

        \underline{underlined words}


    \subsection{Colorful Outputs}
    This part demonstrates the powerful capacities of \LaTeX to output colors. 
    Note that this only happens after using \texttt{usepackage\{color\}} 

    {\color{red}flame} 

    {\color{blue}water} 

    \colorbox{black}{\color{white}background color} 

    \colorbox{yellow}{\color{magenta}Lakers suck} 

    \subsection{Font Size}
    normal size words

    \noindent{\tiny tiny words}

    \noindent{\scriptsize scriptsize words}

    \noindent{\footnotesize footnotesize words}

    \noindent{\small small words}

    \noindent{\large large words}

    \noindent{\Large Large words}

    \noindent{\LARGE LARGE words}

    \noindent{\huge huge words}

\section{Enumerate \& Itemize}
    \subsection{Paragraph indent}
    \LaTeX 's default setting is that
    at the start of the first paragraph in 
    every section, there would be no indent.
    Every other paragraph aside from those ones would have 
    indent as default. 

    If you don't want any indent in a specific paragraph,
    adding the command \texttt{$\backslash$noindent} before 
    the intended paragraph would help.

    If you want all of your paragraphs to be of noindent
    use this command:
    
    \texttt{$\backslash$setlength\{$\backslash$parindent\}\{0pt\}}

    somewhere in the .tex file 
    and all the paragraphs following this cmd would have no indent
    \subsection{Enumerate}
    LaTeX Supports two kinds of Lists:

    A Sorted List(enumerate) and An Unsorted List(itemize)

    The elements in the list can be defined as $\backslash$item and Lists can have Sublists 
        \begin{enumerate}
        \item First thing
        \item Second thing
        \begin{itemize}
        \item A sub-thing
        \item Another sub-thing
        \end{itemize}
        \item Third thing
        \end{enumerate}
    We could use [parameter] to change the representation
    of items in an Unsorted Lists
    For instance,$\backslash$item[-] will use a slash as the 
    new representation
    You could even use a Word for representation,e.g. $\backslash$item[One]
        \begin{itemize}
        \item[-] First thing
        \item[+] Second thing
        \begin{itemize}
        \item[aaa] A sub-thing
        \item[bbb] Another sub-thing
        \end{itemize}
        \item[Q] Third thing
        \end{itemize}


    \subsection{Itemize}
\section{Special Symbols}
\subsection{Spacing and Explanatory Notes}
We use \% to create a explanatory note within the same line
, all characters after this \% would be ignored
until the next line begins.


For example:

\noindent\textit{
It is a truth universally acknowledged \% Note comic irony
\\
in the very first sentence
\\
, that a single man in possession of a good fortune, must
\\
be in want of a wife.
}

would come out as:

\noindent\textit{
It is a truth universally acknowledged% Note comic irony
\\
in the very first sentence
\\
, that a single man in possession of a good fortune, must
\\
be in want of a wife.}

In LaTeX ,multiple consecutive spaces would be seen as one.

Multiple consecutive empty lines would also be seen as one.

The main objective for using an empty line is to start a new paragraph.

In most circumstances, LaTeX ignores empty lines and other symbols representing spacing,
Note that two backslashs($\backslash\backslash$)could be used to switch to
a new line.

If you want to add spaces in your doc, 
you could use the cmd: \\$\backslash$vaspace\{...\}
In this way, we can create vertical spaces,where the 
height of the space can be assigned.

For instance, $\backslash$vspace\{12pt\} will create a space
,which has the height of a character with 12pt in height

\subsection{Special Symbols}

These are considered Special symbols in \LaTeX:

\textbf{\# \$ \% \^{} \& \_ \{ \} \~{} \textbackslash} 

You could refer to the \LaTeX version of this document to see 
how these symbols were typed in.

Note that when using symbols: \^{} and \~{} ,
we need to push a pair of\{\} right after using these symbols.
Otherwise, they would be infered as hats of characters that follows,
just like \textbackslash\^{} e would be interpreted as
\^ e.

Also Note that,backslash can NOT be typed in by simply adding Another
backslash before it (or else it would do the line wrapping thing)
The right way is to use the command:
 \textbackslash textbackslash to substitute it.

Try typing the following sentence just to get a hang of what I mean:

\textit{ Item \#A\textbackslash642 costs \$8 \& is sold
at a \^{} 10\% profit.}
\section{Graphs}
\subsection{Tabular}
The command \textbackslash tabular is used to typeset tabulars.
LaTeX's default is that there are no horizontal or vertical 
seperation lines, so it you need those, you'll have to set it manually.
LaTeX will automatically assign the width of the tabular based on it's content.
use:

\textbackslash begin\{tabular\}\{\dots\}

\noindent{to create a tabular;}

The dots should be replaced by the following:
\begin{itemize}
    \item \textbf{l} :the column will align to the left
    \item \textbf{r} :the column will align to the right
    \item \textbf{c} :the column will align to the center
    \item \textbf{|} :stands for a vertical line
\end{itemize}

For instance,\{lll\} will create a tabular of three columns,
aligned to the left,and no vertical seperation lines.

While \{|l|l|r|\} will generate a three-column tabular,
with the first two columns aligned to the left
and the last column aligned to the right,
there would be vertical seperation lines between columns

Insert the data of the tabular after: \textbackslash begin\{tabular\}\{\dots\}
\begin{itemize}
    \item \textbf{\&} :to seperate columns
    \item \textbf{\textbackslash\textbackslash} :to switch rows
    \item \textbf{\textbackslash hline} :to insert a 
    horizontal line that will cross all columns
    \item \textbf{\textbackslash cline\{1-2\}} :to add a horizontal seperation
    line between row 1 and row 2

\end{itemize}

Lastly,remember to use \textbackslash end \{tabular\} to end the tabular.

Here are some examples(You may wanna use 
them to practice using tabulars yourself):

    \begin{tabular}{|l|l|}
    Apples & Green \\
    Strawberries & Red \\
    Orange & Orange\\
    \end{tabular}
    \hspace{20pt}
    \begin{tabular}{r|c}
    Apples & Green\\
    \hline 
    Strawberries & Red \\
    \cline{1-1}
    Oranges & Orange \\
    \end{tabular}

    \vspace{12pt}
    \begin{tabular}{|r|l|}
    \hline
    8 & here's \\
    \cline{2-2}
    86 & stuff\\
    \hline \hline 
    2008 & now \\
    \hline 
    \end{tabular}
    \hspace{20pt}
    \begin{tabular}{l|r|r}
        Item & Quantity & Price(\$) \\
        \hline
        Nails & 500 & 0.34 \\
        Wooden boards & 100 & 4.00 \\
        Bricks & 240 & 11.50
    \end{tabular}

    \vspace{20pt}
    \begin{tabular}{|l|c|c|c|}
        \hline
        City & 2006 & 2007 & 2008 \\
        \hline
        London & 45789 & 46551 & 51298 \\
        Berlin & 34549 & 32543 & 29870 \\
        Paris & 49835 & 51009 & 51970 \\
        \hline
    \end{tabular}
    \subsection{Image}
    In this subsection, we are going to introduce
    how to insert graphics in \LaTeX.

    Here, we need to include the \textbf{graphicx} package.
    Note thtat the picture should be in the format
    of PDF,PNG,JPEG or GIF. The following code will
    insert a image named "myimage".

    \textbackslash begin\{figure\}[h]

    \textbackslash centering

    \textbackslash includegraphics[width=1\textbackslash textwidth]\{myimage\}

    \textbackslash caption\{Here is my image\}

    \textbackslash label\{image-myimage\}

    \textbackslash end\{figure\}

    \noindent{[h] represents the paramenter for 
    the location of the picture,\\
    h represents placing the image right \textbf{HERE}
    (if there is enough space)\\
    There are other options:
    \\
    \textbf{t} :represents placing the image on the
    \textbf{TOP} of the page;\\
    \textbf{b} :represents placing the image on the
    \textbf{BOTTOM} of the page;\\
    \textbf{p} :represents placing the image on a
    \textbf{NEW PAGE};\\
    You could also add a ! parameter to assign
    the picture to a specific place mandatorily.\\
    (Although in this way, the results might be horrible)}

    \textbackslash centering will place the picture at the center of the page.
    without this command, LaTeX's default will align the picture to the left.
    This command is really useful because it will also
    assigned the title of the picture to the middle.

    The command \textbackslash includegraphics\{...\} 
    can add the picture into your document automatically.
    Note that the picture should be placed under the same 
    directory with the Tex file.

    [width=1\textbackslash textwidth] is an optional parameter,
    It assigns the width of the picture to be the same with the
    text. Width can also be set using cm.
    We could also use [scale=0.5] to shrink the scale of the picture.

    \textbackslash caption\{...\} defines the title(caption) of the picture. 
    By using that,\\LaTeX will assign a serial number starting with "Figure".
    Afterwards, you could use  \textbackslash listoffigures to generate a 
    directory of all the graphs.

    \textbackslash label\{...\} generates a label which you can refer to.

    REMARK: Don't forget to use the command
    
    \hspace{20pt}{\textbackslash usepackage\{graphicx\}}

    before using the functions above(LoL).

    Here is an example:
    \begin{figure}[h]
    \centering
    \includegraphics[width=0.5\textwidth]{demo.png}
    \caption{Demo Fig.}
    \label{image-demo image}
    \end{figure}

\section{Formulas} 
\subsection{Inserting Formulas}
There are mainly four ways to insert Formulas
in LaTeX files.
\begin{enumerate}
    \item To create a formula within the same line
    use \textbf{\$...\$} \\
    Other ways that exists include:
    \textbf{
    \begin{itemize}
        \item \textbackslash( ... \textbackslash)
        \item \textbackslash begin\{math\}... \textbackslash end\{math\}
    \end{itemize}}
    For example: $B^0(X_0,\delta) = \{x\in \mathbb{R}^n | 0< ||X-X_0||<\delta\}$
    \item To create a formula that starts off with a new line,
    use \$\$...\$\$ \\
    For example: $$\aleph_0 ^ {\aleph_0} = \aleph_1$$
    \item To create equations with numbers, use:\\
    \textbf{\textbackslash begin \{equation\} \\ ... \\ \textbackslash end \{equation\}}
    \\For example:
    \begin{equation}
        f(X)=f(X_0)+Jf(X_0)\Delta X + \frac{1}{2!}(\Delta X)^T
        H_{(X_0+\theta \Delta X)} \Delta X
    \end{equation}
    \item If we want to take the equations to a next level:
    A series of equations, we might want to start using the command:
    \\\textbf{\textbackslash begin \{eqnarray\} \\ ... \\ \textbackslash end \{eqnarray\}}
    \\For example:
    \begin{eqnarray}
        \int_{A^{'}}^{A^{''}} f(x,y)g(x,y) dx =
        g(A^{'},y) \int_{A^{'}}^{\xi} f(x,y) dx +
        g(A^{''},y)\int_{\xi} ^{A^{''}} f(x,y) dx \\
       I(y)=\int^{\beta (y)}_{\alpha (y)}f(x,y)dx \\
        \frac{d}{dy}I_{y}=\int_{\alpha(y)}^{\beta(y)}[\frac{\partial f}{\partial y}(x,y)]dx+
        f(\beta(y),y)\beta^{'}{(y)}-f(\alpha(y),y)\alpha^{'}{(y)}
    \end{eqnarray}
    \\We can use \textbf{\& ... \&} between different lines to align them
    in a way that the dots in different lines are aligned together.
\end{enumerate}
\subsection{Special Symbols}
There are literally loads of Symbols and Letters that would 
take up dozens of papers just to list them all on the script.

Therefore, it is my firm belief that the best way to learn language
it by putting it in real-life scenarios and grab what you need 
and add them to your inventory.

The following is a great website where one could find reference
to most commonly-used math symbols and such.

http://www.uinio.com/Math/LaTex/

So what we are covering here might just as well be a beginning of
what a typical -Tex users would get familiar with.

\begin{enumerate}
    \item \textbackslash frac\{...\}\{...\}
    \\ This is used to represent fractions
    \\ For instance: $\frac{x}{\ln x}$
    \item \textbackslash sqrt[...]\{...\}
    \\ This is used for squareroots.
    \\ For example: $\sqrt[3]{\pi}$
    \item \textbackslash sum\_{}\{\}\^{}\{\}
     \textbackslash int\_{}\{\}\^{}\{\}
     \textbackslash prod\_{}\{\}\^{}\{\}
    \\ These stand for adding discrete stuff up; 
    adding continous stuff up(to be specific under Riemann's condition);
    and multiplying discrete stuff up;
    \\Examples:
    \begin{eqnarray}
        \sum_{i=0}^{10}a_{i}\\
    \int_{0}^{a}\Gamma(t)dt\\
        \prod_{i=1}^{m}b_{i}
    \end{eqnarray}
\end{enumerate}

After a while, you will find yourself more and more used to the Greek
 alphabet as well as some Heubric characters. 

 \noindent {REMARK: $\xi$ would be printed using the command \textbackslash xi (At least in China
 This character is usually mispronounced as well as written in very diverse ways lol)}

 However, if you happen to use vscode+ the extension: LaTeX Workshop
 to edit .tex files, you'll find a bar named {\scriptsize{"SNIPPET VIEW"}} on the 
 left hand side, where you could possibly find the characters that
 you might be looking for.
\section{References}
\subsection{Intro}
LaTeX can easily deal with insert reference files and directories.
In this section, we are going to explore how LaTeX does this
by storing the reference files in BibTeX. 
\subsection{BibTeX file format}

BibTeX should include all the files you want to refer to.
It's suffix is .bib.

The name of the .bib file should be set to be identical with your .tex file.

.bib is a document file in which you may wanna insert your reference articles in this format:

\noindent@ article\{\\
    Transformer\_{}Model,\\
    Author = \{A Vaswani,N Shazeer,N Parmar,J Uszkoreit,L Jones,AN Gomez,L Kaiser and I Polosukhin\},\\
    Title = \{Attention \{I\}s \{A\}ll \{Y\}ou \{N\}eed\},\\
    Journal = \{arXiv\},\\
    Volume = \{50\},\\
    Pages = \{9-19\},\\
    Year = \{2017\} \\
\}

With every referenced doc, we should first define it's 
reference type.
The demo is in @article type, other types include 
@book, @incollection for citing chapters in a book,
@inproceedings to cite conference papers.

After that, we write down a citation key.
Please make sure each referenced doc uses a different key.
You can name it whatever you want, but it's better to settle down
with a certain kind of format.

The next few lines should contain information about the file:

The general format is:

\texttt{Field name = {field contents}}


For Capitalized character, use braces to cover them.
BibTeX will automatically de-Capitalized all characters aside from the first char in the title.

You could write BibTeX files manually, but there are also software that generates these files for you.

\subsection{Inserting a List of references}

Use the following code to generate this list right at the spot 
in the .tex file.

\textbackslash bibliographystyle\{plain\}

\textbackslash bibliography\{references\}

Write the cited articles in references.bib

\subsection{Reference Annotation}

Use \textbackslash cite\{citationkey\} to generate an annotation at the intended spot.
If you don't want to insert an annotation in the main body, yet still
want the reference list to display this citation,
use the command \textbackslash nocite\{citationkey\}.

To also include the page information of the reference article,use

\textbackslash cite[p. 215]\{citationkay\}.

Use "," to seperate multiple cited articles. For example:

\textbackslash cite\{citation01,citation02,citation03\}

\subsection{Citation Format}
\begin{enumerate}
    \item Citing by Numbers\\
    \LaTeX includes multiple ways to use numbers
    to cite reference articles:
    \begin{itemize}
        \item Plain\\Uses [ num ] as the format.\\
        For instance,[1].
        The article's authors will be listed based on the
        Lead Author's name's dictionary order. Every Author's name
        would be written down in full name.
        \item Abbrv\\Same with Plain but the authors' names would be abbreviated.
        \item Unsrt\\Same with Plain but the order of the cited articles
        will be based on the sequence the main body cites them.
        \item Alpha\\Same with Plain but the annotation would be in
        the format of [NameAbbrev+num].\\For example:[Ker10]
    \end{itemize}
    \item Citing by Date
    
    If you want to use the format of Name+Date, include the package "natbib".

    It uses the command: \textbackslash citep\{...\} to generate the desired format of citation
    for instance [Koppe,2010]. Use\textbackslash citet\{...\} to only put the year in the [...] part,
    For example, Koppe [2010].

    The Natbib package has three formats:
    
    \textbf{plainnat,abbrvnat \& unsrtnat}
    
    They are identical to \textbf{plain,abbrv \& unsrt}.

    \item Other forms of citation
   
    If you plan on using different citation formats, create a .bst file
    under the same directory, cite this format by using the command:

    \textbackslash bibliographystyle\{...\}

\end{enumerate}

\section{Conclusion}
Let's wrap it up. In this document, we learned the basics of
using LaTeX to get the desired output of our files. We focused
on Font-Related issues; Creating Paragraph and lists(both sorted and unsorted)
 as well as special symbols,graphs,formulas and references. 

Remember? This is the output of reference by using label which we discussed in the second section:

Referring to \ref{sec1} on page \pageref{sec1}

If you do remember, congrats! If you don't, no worries, just 
go over it a few more times. As we all know, language is no barrier
for the sharp-witted, data structures \& algorithms, those are stuff
 that really takes a bit of talent to learn. Learning \LaTeX? It's just
 the kind of technical skill that only takes time to excel. Good luck 
 on your journey with LaTeX as well as other journeys in life. May the Force 
 be with you.

I'll leave you here with a "itemize" of Further Reading:
\begin{itemize}
    \item \textbf{OI-Wiki} https://oi-wiki.org/tools/latex/
    \item \textbf{The \TeX Book} \textit{Donald Knuth}
    \item \textbf{\LaTeX Project}  http://www.latex-project.org/
    \item \textbf{\LaTeX WikiBook}  http://en.wikibooks.org/wiki/LaTeX/ 
\end{itemize}

Many thanks for OIWiki's LaTeX learning webpage, which I used for reference
(to a great extent lol), the github repo for this proj is at 

\listoffigures
\end{document}